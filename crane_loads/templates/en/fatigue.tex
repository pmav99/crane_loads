\section{Fatigue loads}

To calculate the effects of fatigue an equivalent fatigue load is used. For its
calculation  characteristic values of the loads of the crane are used (without
dynamic coefficients). So the maximum characteristic value of the load wheel $Q_{max,i}$
is:
\begin{align*}
    \sum{Q_{max,i}} &= \dfrac{G_{cr}}{2} + \dfrac{(G_{tr} + Q_h) \cdot (L - a)}{L} = {{ "%.3f" % SQmax_i }} \text{ KN} \\
    Q_{max,i}       &= \dfrac{\displaystyle\sum{Q_e}}{2}                           = {{ "%.3f" % Qmax_i }}  \text{ KN}
\end{align*}

The dynamic fatigue factor follows:
\begin{equation*}
    φ_{fat} = \max \left( \dfrac{1+φ_1}{2}\ ; \ \dfrac{1+φ_2}{2} \right) {{ "%.3f" % vfat }}
\end{equation*}

Factors $λ_σ$ και $λ_τ$ depend on the fatigue class of the crane. According to
Tab.2.12 of EC1991-3:2006 for fatigue class $S3$ it holds that:
\begin{align*}
    λ_σ &= {{ "%.3f" % lfat_s }}\\
    λ_τ &= {{ "%.3f" % lfat_t }}
\end{align*}

Finally the wheel load for fatigue design is:
\begin{align*}
    Q_{e,σ} &= φ_{fat} \cdot λ_σ \cdot Q_{max,i} = {{ "%.3f" % Qes }}\\
    Q_{e,τ} &= φ_{fat} \cdot λ_τ \cdot Q_{max,i} = {{ "%.3f" % Qet }}
\end{align*}
