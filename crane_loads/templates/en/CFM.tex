\subsection{Οριζόντια φορτία λόγω παράγωγης κίνησης της γερανογέφυρας (συνδυασμός 5)}

Κατά την κίνηση της η γερανογέφυρα ακολουθεί μια λοξή πορεία ως προς τις τροχιές της. Το γεγονός
αυτό έχει ως συνέπεια τη δημιουργία των εξής δυνάμεων:

\begin{itemize}
    \item της οδηγούσας δύναμης $S$ που ασκείται στη θέση των μέσων καθοδήγησης, και
    \item των εγκάρσιων ως προς τις τροχιές δυνάμεων που ασκούνται σε κάθε τροχό.
\end{itemize}

\begin{align*}
    \sum{Q_r} &= \displaystyle\sum{Q_{r,max}} + \displaystyle\sum{Q_{r,(max)}} = {{ "%.3f" % SQr }} \text{ KN} \\
    \xi_1     &= \dfrac{\displaystyle\sum{Q_{r,max}}}{\displaystyle\sum{Q_{r,max}} + \displaystyle\sum{Q_{r,(max)}}} = {{ "%.3f" % ksi1 }} \\
    \xi_2     &= 1 - \xi1 = {{ "%.3f" % ksi2 }}\\
    \alpha    &= {{ "%.4f" % a_rad }} \text{ rad} \\
    f         &= 0.3 \cdot \left(1 - e^{-250 \cdot \alpha}\right) = {{ "%.3f" % f }}
\end{align*}

Αφού το σύστημα κύλισης της γερανογέφυρας είναι CFM έχουμε:

\begin{align*}
    h                &= \dfrac{m \cdot \xi_1 \cdot L^2 + \sum{e_i^2}}{\sum{e_i}} = {{ "%.3f" % h }} \\
    \lambda_s        &= 1 - \dfrac{\sum{e_i}}{n \cdot h} =  {{ "%.3f" % l_s }} \\
    \lambda_{S,11,L} &= \dfrac{\xi_1 \cdot \xi_2 \cdot L}{n \cdot h}= {{ "%.3f" % l_s11L }} \\
    \lambda_{S,12,L} &= \dfrac{\xi_1 \cdot \xi_2 \cdot L}{n \cdot h}= {{ "%.3f" % l_s12L }} \\
    \lambda_{S,21,L} &= \dfrac{\xi_1 \cdot \xi_2 \cdot L}{n \cdot h}= {{ "%.3f" % l_s21L }} \\
    \lambda_{S,22,L} &= \dfrac{\xi_1 \cdot \xi_2 \cdot L}{n \cdot h}= {{ "%.3f" % l_s22L }} \\
    \lambda_{S,11,T} &= \dfrac{\xi_2}{n} \cdot \left(1 - \dfrac{e_1}{h}\right) = {{ "%.3f" % l_s11T }} \\
    \lambda_{S,12,T} &= \dfrac{\xi_2}{n} \cdot \left(1 - \dfrac{e_2}{h}\right) = {{ "%.3f" % l_s12T }} \\
    \lambda_{S,21,T} &= {{ "%.3f" % l_s21T }} \\
    \lambda_{S,22,T} &= {{ "%.3f" % l_s22T }}
\end{align*}

Άρα οι ασκούμενες δυνάμεις είναι:

\begin{align*}
    S                &= f \cdot \lambda_s \cdot \sum{Q_r} = {{ "%.3f" % S }} \text{ KN} \\
    H_{S,11,L}       &= f \cdot \lambda_{s,11,L} \cdot \sum{Q_r} = {{ "%.3f" % H_s11L }} \text{ KN} \\
    H_{S,12,L}       &= f \cdot \lambda_{s,12,L} \cdot \sum{Q_r} = {{ "%.3f" % H_s12L }} \text{ KN} \\
    H_{S,21,L}       &= f \cdot \lambda_{s,21,L} \cdot \sum{Q_r} = {{ "%.3f" % H_s21L }} \text{ KN} \\
    H_{S,22,L}       &= f \cdot \lambda_{s,22,L} \cdot \sum{Q_r} = {{ "%.3f" % H_s22L }} \text{ KN} \\
    H_{S,11,T}       &= f \cdot \lambda_{s,11,T} \cdot \sum{Q_r} = {{ "%.3f" % H_s11T }} \text{ KN} \\
    H_{S,12,T}       &= f \cdot \lambda_{s,12,T} \cdot \sum{Q_r} = {{ "%.3f" % H_s12T }} \text{ KN} \\
    H_{S,21,T}       &= f \cdot \lambda_{s,21,T} \cdot \sum{Q_r} = {{ "%.3f" % H_s21T }} \text{ KN} \\
    H_{S,22,T}       &= f \cdot \lambda_{s,22,T} \cdot \sum{Q_r} = {{ "%.3f" % H_s22T }} \text{ KN}
\end{align*}

Τελικά στη θέση του προπορευόμενου ζεύγους τροχών έχουμε:

\begin{align*}
    H_{S,1T} &= H_{S,11,T} - S = {{ "%.3f" % H_s1T }} \text{ KN} \\
    H_{S,2T} &= H_{S,21,T} = {{ "%.3f" % H_s21T }} \text{ KN}
\end{align*}
