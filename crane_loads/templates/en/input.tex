\section{Input}
\subsection{General}

Factors $φ_{2,min}$ and $β_2$ depent on the hoisting class and
are calculated based on Tab.~2.5 of EC1991-3:2006.

\begin{tabular}{llcr}
    Fatigue class                       &$=$ {{ FC }} \\
    scrolling system                    &$=$ {{ RT }}\\
    Wheels per axle    $n_r$            &$=$ ${{ nr }} $ \\
    Individual driving wheels $m_w$     &$=$ ${{ mw }}$ \\
    Friction factor $\mu$               &$=$ ${{ "%.3f" % mf }}$ \\
    Hoisting class                      &$=$ {{ HC }} \\
    $φ_{2,min}$                         &$=$ ${{ v2_min }}$ \\
    $β_2$                               &$=$ ${{ b2 }}$ \\
    Hoisting speed  $V_h$               &$=$ ${{ "%.3f" % vh }}$ m/sec
\end{tabular}

\subsection{Loads}
The values ​​of loads are characteristic ones.

\begin{tabular}{llcrl}
    Self-weight of crane gantry     & $G_{cr}$  &$=$ &${{ "%.2f" % Gcr }}$ &KN \\
    Self-weight of trolley          & $G_{tr}$  &$=$ &${{ "%.2f" % Gtr }}$ &KN \\
    Total self-weight               & $G_{tot}$ &$=$ &${{ "%.2f" % Gtot }}$ &KN \\
    Safe working load               & $Q_{nom}$ &$=$ &${{ "%.2f" % Qr_nom }}$ &KN
\end{tabular}

\subsection{Geometry}
The crane's geometry is as follows:

\begin{tabular}{llcrl}
    Crane span                                      &$L$          &$=$ &${{ "%.3f" % L }}$ &m \\
    Minimum distance of hoist from rail-beams       &$e_{min}$    &$=$ &${{ "%.3f" % e_min }}$ &m \\
    Wheel distance upon rail-beam                   &$a$          &$=$ &${{ "%.3f" % a }}$ &m \\
    Wheel distance $1$ from guidance means          &$e_1$        &$=$ &${{ "%.3f" % e1 }}$ &m \\
    Wheel distance $2$ from guidance means          &$e_2$        &$=$ &${{ "%.3f" % e2 }}$ &m \\
    Skewing angle                                   &$α$          &$=$ &${{ "%.5f" % a_rad }}$ &rad
\end{tabular}

\section{Dynamic factors}
Dynamic factors are calculated according to EΝ1991-3:2006. More specifically:

\begin{itemize}
    \item Factor $ φ_1 $ applies to permanent loads and represents the vibrations of the crane.
    \item Factor $ φ_2 $ applies to safe working load (SWL) and its value varies depending on the lifting class and lifting speed.
    \item Factor $ φ_3 $ depends on the way that the hoist load is lifted.  For safety may be considered equal to one.
    \item Factor $ φ_4 $, taken into account that the rail tolerances specified in EN1993-6: 2007, can be taken equal to unity.
    \item Factor $ φ_5 $ depends on the way in which forces in crane gantry fluctuate.
    \item Factor $ φ_6 $ depends on the pattern of the test load and the size of the test load.
\end{itemize}

\begin{tabular}{llcr}
    $φ_1$ &$= {{ v1 }} $ \\
    $φ_2$ &$= φ_{2,min} + β_2 \cdot V_h = {{ v2_min }} + {{ b2 }} \cdot {{ "%.3f" % vh }} = {{ "%.3f" % v2 }} $ \\
    $φ_3$ &$= {{ v3 }} $ \\
    $φ_4$ &$= {{ v4 }} $ \\
    $φ_5$ &$= {{ v5 }} $ \\
    $φ_6$ &$= {{ v6 }} $
\end{tabular}
