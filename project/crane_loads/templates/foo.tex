\documentclass[a4paper,10pt, final, oneside, fleqn, onecolumn]{article}	%fleqn

\usepackage{hyperref}

\usepackage{amsmath}
\usepackage[math-style=TeX]{unicode-math}
\setmathfont{XITS Math}

\usepackage{xltxtra}
\usepackage{xgreek}
\setmainfont[%
    Kerning=On,
    Mapping=tex-text,
    UprightFont = Minion Pro,
    ItalicFont = Minion Pro Italic,
    SlantedFont = Minion Pro Italic,
    BoldFont = Minion Pro Semibold,
    BoldItalicFont = Minion Pro Semibold Italic,
    BoldSlantedFont = Minion Pro Semibold Italic,
    SmallCapsFont = Minion Pro,
        ]{Minion Pro}
\usepackage[hmargin=2.5cm,vmargin=3cm]{geometry}

\usepackage{booktabs}
\usepackage{multirow}
\usepackage{rotating}

\usepackage{fancyhdr}
\pagestyle{fancy}
\lhead{}
\chead{\bfseries Crane Loads}
\rhead{}
\lfoot{\textbf{XXXXXXX}}
\cfoot{\thepage}
\rfoot{\textbf{XXXXXXXX}}
\renewcommand{\headrulewidth}{0.4pt}
\renewcommand{\footrulewidth}{0.4pt}

\widowpenalty=1000
\clubpenalty=1000

\setlength{\tabcolsep}{2pt}
\begin{document}

\section{Input}
\subsection{General}

Factors $φ_{2,min}$ and $β_2$ depent on the hoisting class and
are calculated based on Tab.~2.5 of EC1991-3:2006.

\begin{tabular}{llcr}
    Fatigue class                       &$=$ S3 \\
    scrolling system                    &$=$ IFF\\
    Wheels per axle    $n_r$            &$=$ $2 $ \\
    Individual driving wheels $m_w$     &$=$ $2$ \\
    Friction factor $\mu$               &$=$ $0.2$ \\
    Hoisting class                      &$=$ HC2 \\
    $φ_{2,min}$                         &$=$ $1.1$ \\
    $β_2$                               &$=$ $0.34$ \\
    Hoisting speed  $V_h$               &$=$ $0.083$ m/sec
\end{tabular}

\subsection{Loads}
The values ​​of loads are characteristic ones.

\begin{tabular}{llcr}
    Self-weight of crane gantry     & $G_{cr}$  &$=$ &$89.28$ KN \\
    Self-weight of trolley          & $G_{tr}$  &$=$ &$11.84$ KN \\
    Total self-weight               & $G_{tot}$ &$=$ &$101.12$ KN \\
    Safe working load               & $Q_{nom}$ &$=$ &$80$ KN
\end{tabular}

\subsection{Geometry}
Τα γεωμετρικά χαρακτηριστικά της γερανογέφυρας είναι:

\begin{tabular}{llcl}
    Crane span                                      &$L$          &$=$ &$18$ m \\
    Minimum distance of hoist from rail-beams       &$e_{min}$    &$=$ &$1.51$ m \\
    Wheel distance upon rail-beam                   &$a$          &$=$ &$4$ m \\
    Wheel distance $1$ from guidance means          &$e_1$        &$=$ &$0$ m \\
    Wheel distance $2$ from guidance means          &$e_2$        &$=$ &$4$ m \\
    Skewing angle                                   &$α$          &$=$ &$0.0015$ rad
\end{tabular}

\section{Dynamic factors}
Dynamic factors are calculated according to EΝ1991-3:2006.

Factor $ f_1 $ applies to permanent loads and represents the vibrations of the crane. 
Factor $ f_2 $ applies to safe working load (SWL) and its value varies depending on the lifting class 
and lifting speed. Factor $ f_3 $ depends on the way that the hoist load is lifted. For safety may be 
considered equal to one. Factor $ f_4 $, taken into account that the rail tolerances specified in 
EN1993-6: 2007, can be taken equal to unity. Factor $ f_5 $ depends on the way in which forces in 
crane gantry fluctuate. Factor $ f_6 $ depends on the pattern of the test load and the size of the 
test load.

\begin{tabular}{llcr}
    $φ_1$ &$= 1.1 $ \\
    $φ_2$ &$= φ_{2,min} + β_2 \cdot V_h = 1.1 + 0.34 \cdot 0.083 = 1.128 $ \\
    $φ_3$ &$= 1 $ \\
    $φ_4$ &$= 1 $ \\
    $φ_5$ &$= 1.5 $ \\
    $φ_6$ &$= 1 $
\end{tabular}

\section{Loads}

In overall there are 10 load combinations. The calculation of the loads takes place by multiplying 
the characteristic values of the self-weights and safe working load (SWL) with the according dynamic 
factors $φ_i$, as specified in Tab.~2.2 of EC1991-3:2006. then, and for each load combination, the max 
(loaded crane) and min (unloaded crane) wheel loads of the crane girder are calculated. 

\subsection{Vertical forces}
\subsubsection{1st load combination}

Dynamic loading:
\begin{align*}
    G_{cr,φ} &= G_{cr} \cdot φ_1  = 98.208 \text{ KN} \\
    G_{tr,φ} &= G_{tr} \cdot φ_1  = 13.024 \text{ KN} \\
    Qh       &= Q_{nom} \cdot φ_2 = 90.267   \text{ KN}
\end{align*}

Unloaded crane:
\begin{align*}
    \sum{Q_{r,min}}   &= \dfrac{G_{cr,φ}}{2} + \dfrac{G_{tr,φ} \cdot (L - a)}{L} = 61.035 \text{ KN} \\
    Q_{r,min}         &= \dfrac{\displaystyle\sum{Q_{r,min}}}{2}                 = 30.518  \text{ KN} \\
    \sum{Q_{r,(min)}} &= \dfrac{G_{cr,φ}}{2} + \dfrac{G_{tr,φ} \cdot a}{L}       = 50.197 \text{ KN} \\
    Q_{r,(min)}       &= \dfrac{\displaystyle\sum{Q_{r,(min)}}}{2}               = 25.098  \text{ KN}
\end{align*}

Loaded crane:
\begin{align*}
    \sum{Q_{r,\max}}     &= \dfrac{G_{cr,φ}}{2} + \dfrac{(G_{tr,φ} + Qh) \cdot (L - a)}{L} = 143.730 \text{ KN} \\
    Q_{r,\max}           &= \dfrac{\displaystyle\sum{Q_{r,\max}}}{2}                       = 71.865  \text{ KN} \\
    \sum{Q_{r,(\max)}}   &= \dfrac{G_{cr,φ}}{2} + \dfrac{(G_{tr,φ} + Qh) \cdot a}{L}       = 57.769 \text{ KN} \\
    Q_{r,(\max)}         &= \dfrac{\displaystyle\sum{Q_{r,(min)}}}{2}                      = 28.884 \text{ KN}
\end{align*}

\subsubsection{2nd load combination}

Dynamic loading
\begin{align*}
    G_{cr,φ} &= G_{cr} \cdot φ_1  = 98.208 \text{ KN} \\
    G_{tr,φ} &= G_{tr} \cdot φ_1  = 13.024 \text{ KN} \\
    Qh       &= Q_{nom} \cdot φ_3 = 80.000   \text{ KN}
\end{align*}

Unloaded crane:
\begin{align*}
    \sum{Q_{r,min}}   &= \dfrac{G_{cr,φ}}{2} + \dfrac{G_{tr,φ} \cdot (L - a)}{L} = 61.035 \text{ KN} \\
    Q_{r,min}         &= \dfrac{\displaystyle\sum{Q_{r,min}}}{2}                 = 30.518  \text{ KN} \\
    \sum{Q_{r,(min)}} &= \dfrac{G_{cr,φ}}{2} + \dfrac{G_{tr,φ} \cdot a}{L}       = 50.197 \text{ KN} \\
    Q_{r,(min)}       &= \dfrac{\displaystyle\sum{Q_{r,(min)}}}{2}               = 25.098  \text{ KN}
\end{align*}

Loaded crane:
\begin{align*}
    \sum{Q_{r,\max}}     &= \dfrac{G_{cr,φ}}{2} + \dfrac{(G_{tr,φ} + Qh) \cdot (L - a)}{L} = 134.324 \text{ KN} \\
    Q_{r,\max}           &= \dfrac{\displaystyle\sum{Q_{r,\max}}}{2}                       = 67.162  \text{ KN} \\
    \sum{Q_{r,(\max)}}   &= \dfrac{G_{cr,φ}}{2} + \dfrac{(G_{tr,φ} + Qh) \cdot a}{L}       = 56.908 \text{ KN} \\
    Q_{r,(\max)}         &= \dfrac{\displaystyle\sum{Q_{r,(min)}}}{2}                      = 28.454 \text{ KN}
\end{align*}

\subsubsection{3rd load combination}

Dynamic loading
\begin{align*}
    G_{cr,φ} &= G_{cr} \cdot 1.00  = 89.280 \text{ KN} \\
    G_{tr,φ} &= G_{tr} \cdot 1.00  = 11.840 \text{ KN} \\
    Qh       &= Q_{nom} \cdot 0.00 = 0.000   \text{ KN}
\end{align*}

Unloaded crane:
\begin{align*}
    \sum{Q_{r,min}}   &= \dfrac{G_{cr,φ}}{2} + \dfrac{G_{tr,φ} \cdot (L - a)}{L} = 55.487 \text{ KN} \\
    Q_{r,min}         &= \dfrac{\displaystyle\sum{Q_{r,min}}}{2}                 = 27.743  \text{ KN} \\
    \sum{Q_{r,(min)}} &= \dfrac{G_{cr,φ}}{2} + \dfrac{G_{tr,φ} \cdot a}{L}       = 45.633 \text{ KN} \\
    Q_{r,(min)}       &= \dfrac{\displaystyle\sum{Q_{r,(min)}}}{2}               = 22.817  \text{ KN}
\end{align*}

Loaded crane:
\begin{align*}
    \sum{Q_{r,\max}}     &= \dfrac{G_{cr,φ}}{2} + \dfrac{(G_{tr,φ} + Qh) \cdot (L - a)}{L} = 55.487 \text{ KN} \\
    Q_{r,\max}           &= \dfrac{\displaystyle\sum{Q_{r,\max}}}{2}                       = 27.743  \text{ KN} \\
    \sum{Q_{r,(\max)}}   &= \dfrac{G_{cr,φ}}{2} + \dfrac{(G_{tr,φ} + Qh) \cdot a}{L}       = 45.633 \text{ KN} \\
    Q_{r,(\max)}         &= \dfrac{\displaystyle\sum{Q_{r,(min)}}}{2}                      = 22.817 \text{ KN}
\end{align*}

\subsubsection{4th load combination}

Dynamic loading
\begin{align*}
    G_{cr,φ} &= G_{cr} \cdot φ_4  = 89.280 \text{ KN} \\
    G_{tr,φ} &= G_{tr} \cdot φ_4  = 11.840 \text{ KN} \\
    Qh       &= Q_{nom} \cdot φ_4 = 80.000   \text{ KN}
\end{align*}

Unloaded crane:
\begin{align*}
    \sum{Q_{r,min}}   &= \dfrac{G_{cr,φ}}{2} + \dfrac{G_{tr,φ} \cdot (L - a)}{L} = 55.487 \text{ KN} \\
    Q_{r,min}         &= \dfrac{\displaystyle\sum{Q_{r,min}}}{2}                 = 27.743  \text{ KN} \\
    \sum{Q_{r,(min)}} &= \dfrac{G_{cr,φ}}{2} + \dfrac{G_{tr,φ} \cdot a}{L}       = 45.633 \text{ KN} \\
    Q_{r,(min)}       &= \dfrac{\displaystyle\sum{Q_{r,(min)}}}{2}               = 22.817  \text{ KN}
\end{align*}

Loaded crane:
\begin{align*}
    \sum{Q_{r,\max}}     &= \dfrac{G_{cr,φ}}{2} + \dfrac{(G_{tr,φ} + Qh) \cdot (L - a)}{L} = 128.776 \text{ KN} \\
    Q_{r,\max}           &= \dfrac{\displaystyle\sum{Q_{r,\max}}}{2}                       = 64.388  \text{ KN} \\
    \sum{Q_{r,(\max)}}   &= \dfrac{G_{cr,φ}}{2} + \dfrac{(G_{tr,φ} + Qh) \cdot a}{L}       = 52.344 \text{ KN} \\
    Q_{r,(\max)}         &= \dfrac{\displaystyle\sum{Q_{r,(min)}}}{2}                      = 26.172 \text{ KN}
\end{align*}

\subsubsection{5th load combination}Similar to 4th.

\subsubsection{6th load combination}Similar to 4th.

\subsection{Horizontal forces due to acceleration or deceleration}

The drive force of the crane $ K $ is applied on the axis of motion. This force splits equally
to two forces $ H_L $ along the horizontal track applied on the wheels. However, the 
centre of gravity of the the crane gantry-hoist load system is within $ l_s $ from the axis of 
motion, resulting in a twisting moment generated $ M $. To balance those a pair of opposing 
forces $ H_ {T1} $ - $ H_ {T2} $ is applied in the position of the wheels.


\subsubsection{1st and 2nd load combination}

For the first two load combination it holds that:
\begin{align*}
    K         &= μ \cdot m_w \cdot Q_{r,(min)}  = 10.039 \text{KN} \\
    H_{L,1}   &= \dfrac{φ_5 \cdot K} {n_r} = 7.529 \text{KN} \\
    H_{L,2}   &= \dfrac{φ_5 \cdot K} {n_r} = 7.529 \text{KN} \\
    ξ_1       &= \dfrac{\displaystyle\sum{Q_{r,max}}}{\displaystyle\sum{Q_{r,max} + \displaystyle\sum{Q_{r,(max)}}}} = 0.713 \\
    ξ_2       &= 1 - ξ1 = 0.287 \\
    l_s       &= (ξ_1 - 0.5) \cdot ξ_2 = 3.839 \text{m} \\
    M         &= K \cdot l_s = 38.546 \text{KNm} \\
    H_{T1}    &= φ_5 \cdot ξ_2 \cdot \dfrac{M}{a} = 4.144 \text{KN} \\
    H_{T2}    &= φ_5 \cdot ξ_1 \cdot \dfrac{M}{a} = 10.311 \text{KN}
\end{align*}

\subsubsection{3rd, 4th and 8th load combinations}

For the 3rd, 4th and 8th load combination it holds that:
\begin{align*}
    K         &= μ \cdot m_w \cdot Q_{r,(min)}  = 9.127 \text{KN} \\
    H_{L,1}   &= \dfrac{φ_5 \cdot K} {n_r} = 6.845 \text{KN} \\
    H_{L,2}   &= \dfrac{φ_5 \cdot K} {n_r} = 6.845 \text{KN} \\
    ξ_1       &= \dfrac{\displaystyle\sum{Q_{r,max}}}{\displaystyle\sum{Q_{r,max} + \displaystyle\sum{Q_{r,(max)}}}} = 0.711 \\
    ξ_2       &= 1 - ξ1 = 0.289 \\
    l_s       &= (ξ_1 - 0.5) \cdot ξ_2 = 3.798 \text{m} \\
    M         &= K \cdot l_s = 34.662 \text{KNm} \\
    H_{T1}    &= φ_5 \cdot ξ_2 \cdot \dfrac{M}{a_r} = 3.757 \text{KN} \\
    H_{T2}    &= φ_5 \cdot ξ_1 \cdot \dfrac{M}{a_r} = 9.242 \text{KN}
\end{align*}


\subsection{Horizontal loads caused by skewing of the crane (load combination 5)}

During its motion the crane gantry follows a skewed course in relation to its wheels. This has as an 
effect the development of the following forces:

\begin{itemize}
    \item the driving force $S$ that applies on the guidance means, and
    \item the lateral to the rails forces applied on the wheels.
\end{itemize}

\begin{align*}
    \sum{Q_r} &= \displaystyle\sum{Q_{r,max}} + \displaystyle\sum{Q_{r,(max)}} = 181.120 \text{ KN} \\
    \xi_1     &= \dfrac{\displaystyle\sum{Q_{r,max}}}{\displaystyle\sum{Q_{r,max}} + \displaystyle\sum{Q_{r,(max)}}} = 0.711 \\
    \xi_2     &= 1 - \xi1 = 0.289\\
    \alpha    &= 0.0015 \text{ rad} \\
    f         &= 0.3 \cdot \left(1 - e^{-250 \cdot \alpha}\right) = 0.094
\end{align*}

Since the scrolling system of the crane is IFF:
\begin{align*}
    h                &= \dfrac{m \cdot \xi_1 \cdot \xi_2 \cdot L^2 + \sum{e_i^2}}{\sum{e_i}} = 4.000 \\
    \lambda_s        &= 1 - \dfrac{\sum{e_i}}{n \cdot h} =  0.500 \\
    \lambda_{S,11,L} &= 0.000 \\
    \lambda_{S,12,L} &= 0.000 \\
    \lambda_{S,21,L} &= 0.000 \\
    \lambda_{S,22,L} &= 0.000 \\
    \lambda_{S,11,T} &= \dfrac{\xi_2}{n} \cdot \left(1 - \dfrac{e_1}{h}\right) = 0.145 \\
    \lambda_{S,12,T} &= \dfrac{\xi_2}{n} \cdot \left(1 - \dfrac{e_2}{h}\right) = 0.000 \\
    \lambda_{S,21,T} &= \dfrac{\xi_1}{n} \cdot \left(1 - \dfrac{e_1}{h}\right) = 0.355 \\
    \lambda_{S,22,T} &= \dfrac{\xi_1}{n} \cdot \left(1 - \dfrac{e_2}{h}\right) = 0.000
\end{align*}

Thus, the applied forces are:
\begin{align*} %
    S                &= f \cdot \lambda_s \cdot \sum{Q_r} = 8.496 \text{ KN} \\
    H_{S,11,L}       &= f \cdot \lambda_{s,11,L} \cdot \sum{Q_r} = 0.000 \text{ KN} \\
    H_{S,12,L}       &= f \cdot \lambda_{s,12,L} \cdot \sum{Q_r} = 0.000 \text{ KN} \\
    H_{S,21,L}       &= f \cdot \lambda_{s,21,L} \cdot \sum{Q_r} = 0.000 \text{ KN} \\
    H_{S,22,L}       &= f \cdot \lambda_{s,22,L} \cdot \sum{Q_r} = 0.000 \text{ KN} \\
    H_{S,11,T}       &= f \cdot \lambda_{s,11,T} \cdot \sum{Q_r} = 2.455 \text{ KN} \\
    H_{S,12,T}       &= f \cdot \lambda_{s,12,T} \cdot \sum{Q_r} = 0.000 \text{ KN} \\
    H_{S,21,T}       &= f \cdot \lambda_{s,21,T} \cdot \sum{Q_r} = 6.040 \text{ KN} \\
    H_{S,22,T}       &= f \cdot \lambda_{s,22,T} \cdot \sum{Q_r} = 0.000 \text{ KN}
\end{align*}

Finally, at the position of the leading pair of wheels:
\begin{align*}
    H_{S,1T} &= H_{S,11,T} - S = -6.040 \text{ KN} \\
    H_{S,2T} &= H_{S,21,T} = 6.040 \text{ KN}
\end{align*}

\section{Fatigue loads}

To calculate the effects of fatigue an equivalent fatigue load is used. For its  
calculation  characteristic values of the loads of the crane are used (without 
dynamic coefficients). So the maximum characteristic value of the load wheel $Q_{max,i}$ 
is:

\begin{align*}
    \sum{Q_{max,i}} &= \dfrac{G_{cr}}{2} + \dfrac{(G_{tr} + Q_h) \cdot (L - a)}{L} = 128.776 \text{ KN} \\
    Q_{max,i}       &= \dfrac{\displaystyle\sum{Q_e}}{2}                           = 64.388  \text{ KN}
\end{align*}

The dynamic fatigue factor follows:

\begin{equation*}
    φ_{fat} = \max \left( \dfrac{1+φ_1}{2}\ ; \ \dfrac{1+φ_2}{2} \right) 1.064
\end{equation*}

Factors $λ_σ$ και $λ_τ$ depend on the fatigue class of the crane. According to 
Tab.2.12 of EC1991-3:2006 for fatigue class $S3$ it holds that:

\begin{align*}
    λ_σ &= 0.397\\
    λ_τ &= 0.575
\end{align*}

Finally the wheel load for fatigue design is:

\begin{align*}
    Q_{e,σ} &= φ_{fat} \cdot λ_σ \cdot Q_{max,i} = 27.202\\
    Q_{e,τ} &= φ_{fat} \cdot λ_τ \cdot Q_{max,i} = 39.399
\end{align*}

\section{Aggregated load tables}

The following tables are aggregated loads of both the crane 
serviceability limit state (SLS) and the ultimate limit state (ULS)

\begin{sidewaystable}[htpb]
\begin{center}
\begin{tabular}{ccrrrrrr}
\toprule
\multicolumn{3}{c}{Load combination}                                    & \makebox[1cm][c]{1st}  & \makebox[1cm][c]{2nd}  & \makebox[1cm][c]{3rd}  & \makebox[1cm][c]{4th}  & \makebox[1cm][c]{5th} \\   \midrule
%
\multirow{4}{*}{Vertical Forces}  & \multirow{2}{*}{Self-weight}        & $Q_{r,min}$       & $30.52$   & $30.52$   & $27.74$   & $27.74$   & $27.74$ \\
                                    &                                   & $Q_{r,(min)}$     & $25.10$   & $25.10$   & $22.82$   & $22.82$   & $22.82$ \\ \cmidrule(l){2-8}
                                    & Self-weight +                     & $Q_{r,max}$       & $71.86$   & $67.16$   & \makebox[1cm][c]{-}           & $64.39$   & $64.39$ \\
                                    & hoist load                        & $Q_{r,(max)}$     & $28.88$   & $28.45$   & \makebox[1cm][c]{-}           & $26.17$   & $26.17$ \\ \midrule
%
\multirow{6}{*}{Horizontal Forces}   & \multirow{4}{*}{Acceleration}    & $H_{L1}$    & $7.53$      & $7.53$    & $6.84$    & $6.84$    & \makebox[1cm][c]{-}   \\
                                    &                                   & $H_{L2}$    & $7.53$      & $7.53$    & $6.84$    & $6.84$    & \makebox[1cm][c]{-}   \\
                                    &                                   & $H_{T1}$    & $4.14$    & $4.14$  & $3.76$  & $9.24$  & \makebox[1cm][c]{-}   \\
                                    &                                   & $H_{T2}$    & $10.31$    & $10.31$  & $9.24$  & $9.24$  & \makebox[1cm][c]{-}   \\ \cmidrule(l){2-8}
%
                                    & \multirow{2}{*}{Skewing motion}   & $H_{S,1T}$  & \makebox[1cm][c]{-}  & \makebox[1cm][c]{-}  & \makebox[1cm][c]{-}  & \makebox[1cm][c]{-}  & $-6.04$ \\
                                    &                                   & $H_{S,2T}$  & \makebox[1cm][c]{-}  & \makebox[1cm][c]{-}  & \makebox[1cm][c]{-}  & \makebox[1cm][c]{-}  & $6.04$ \\ \bottomrule
%
\end{tabular}
\end{center}
\caption{Crane loads, in Serviceability Limit State (SLS) ($γ = 1.00$)}
\end{sidewaystable}

\begin{sidewaystable}[htpb]
\begin{center}
\begin{tabular}{ccrrrrrr}
\toprule
\multicolumn{3}{c}{Συνδυασμός φόρτισης}                                                     & \makebox[1cm][c]{1ος}  & \makebox[1cm][c]{2ος}  & \makebox[1cm][c]{3ος}  & \makebox[1cm][c]{4ος}  & \makebox[1cm][c]{5ος} \\   \midrule
%
\multirow{4}{*}{Vertical Forces}  & \multirow{2}{*}{Self-weight}        & $Q_{r,min}$       & $41.20$    & $41.20$    & $37.45$    & $37.45$    & $37.45$ \\
                                    &                                   & $Q_{r,(min)}$     & $33.88$    & $33.88$    & $30.80$    & $30.80$    & $30.80$ \\ \cmidrule(l){2-8}
                                    &  Self-weight +                    & $Q_{r,max}$       & $97.02$    & $90.67$    & \makebox[1cm][c]{-}       & $86.92$    & $86.92$ \\
                                    &  hoist load                       & $Q_{r,(max)}$     & $38.99$    & $38.41$    & \makebox[1cm][c]{-}       & $35.33$    & $35.33$ \\ \midrule
%
\multirow{6}{*}{Horizontal Forces}   & \multirow{4}{*}{Acceleration}       & $H_{L1}$    & $10.16$      & $10.16$    & $9.24$    & $9.24$    & \makebox[1cm][c]{-}   \\
                                    &                                   & $H_{L2}$    & $10.16$      & $10.16$    & $9.24$    & $9.24$    & \makebox[1cm][c]{-}   \\
                                    &                                   & $H_{T1}$    & $5.59$    & $5.59$  & $5.07$  & $5.07$  & \makebox[1cm][c]{-}   \\
                                    &                                   & $H_{T2}$    & $13.92$    & $13.92$  & $12.48$  & $12.48$  & \makebox[1cm][c]{-}   \\ \cmidrule(l){2-8}
%
                                    & \multirow{2}{*}{Skewing motion}   & $H_{S,1T}$  & \makebox[1cm][c]{-}  & \makebox[1cm][c]{-}  & \makebox[1cm][c]{-}  & \makebox[1cm][c]{-}  & $-8.15$ \\
                                    &                                   & $H_{S,2T}$  & \makebox[1cm][c]{-}  & \makebox[1cm][c]{-}  & \makebox[1cm][c]{-}  & \makebox[1cm][c]{-}  & $8.15$ \\ \bottomrule
%
\end{tabular}
\end{center}
\caption{Crane loads, in Ultimate Limit State (ULS) ($γ = 1.00$)}
\end{sidewaystable}

\end{document}