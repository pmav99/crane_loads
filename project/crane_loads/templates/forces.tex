\section{Φορτία}

Συνολικά ορίζονται 10 συνδυασμοί φόρτισης. Ο υπολογισμός των φορτίων γίνεται πολλαπλασιάζοντας τις
χαρακτηριστικές τιμές των των ιδίων βαρών και των ωφέλιμων φορτίων με τους αντίστοιχους δυναμικούς
συντελεστές $φ_i$, όπως ορίζονται στον πίνακα~2.2 του EC1991-3:2006. Στη συνέχεια, και για κάθε
συνδυασμό φόρτισης, υπολογίζονται τα μέγιστα (φορτισμένη γερανογέφυρα) και τα ελάχιστα (αφόρτιστη
γερανογέφυρα) φορτία τροχού της γερανογέφυρας.

\subsection{Κατακόρυφα φορτία}
\subsubsection{1ος συνδυασμός φορτίων}

Δυναμικά φορτία:
\begin{align*}
    G_{cr,φ} &= G_{cr} \cdot φ_1  = {{ "%.3f" % QR1.Gcr_v }} \text{ KN} \\
    G_{tr,φ} &= G_{tr} \cdot φ_1  = {{ "%.3f" % QR1.Gtr_v }} \text{ KN} \\
    Qh       &= Q_{nom} \cdot φ_2 = {{ "%.3f" % QR1.Qh }}   \text{ KN}
\end{align*}

Αφόρτιστη γερανογέφυρα:
\begin{align*}
    \sum{Q_{r,min}}   &= \dfrac{G_{cr,φ}}{2} + \dfrac{G_{tr,φ} \cdot (L - a)}{L} = {{ "%.3f" % QR1.SQr_max }} \text{ KN} \\
    Q_{r,min}         &= \dfrac{\displaystyle\sum{Q_{r,min}}}{2}                 = {{ "%.3f" % QR1.Qr_max }}  \text{ KN} \\
    \sum{Q_{r,(min)}} &= \dfrac{G_{cr,φ}}{2} + \dfrac{G_{tr,φ} \cdot a}{L}       = {{ "%.3f" % QR1.SQr_min }} \text{ KN} \\
    Q_{r,(min)}       &= \dfrac{\displaystyle\sum{Q_{r,(min)}}}{2}               = {{ "%.3f" % QR1.Qr_min }}  \text{ KN}
\end{align*}

Φορτισμένη γερανογέφυρα
\begin{align*}
    \sum{Q_{r,\max}}     &= \dfrac{G_{cr,φ}}{2} + \dfrac{(G_{tr,φ} + Qh) \cdot (L - a)}{L} = {{ "%.3f" % QR1.SQr_MAX }} \text{ KN} \\
    Q_{r,\max}           &= \dfrac{\displaystyle\sum{Q_{r,\max}}}{2}                       = {{ "%.3f" % QR1.Qr_MAX }}  \text{ KN} \\
    \sum{Q_{r,(\max)}}   &= \dfrac{G_{cr,φ}}{2} + \dfrac{(G_{tr,φ} + Qh) \cdot a}{L}       = {{ "%.3f" % QR1.SQr_MIN }} \text{ KN} \\
    Q_{r,(\max)}         &= \dfrac{\displaystyle\sum{Q_{r,(min)}}}{2}                      = {{ "%.3f" % QR1.Qr_MIN }} \text{ KN}
\end{align*}

\subsubsection{2ος συνδυασμός φορτίων}
Δυναμικά  φορτία
\begin{align*}
    G_{cr,φ} &= G_{cr} \cdot φ_1  = {{ "%.3f" % QR2.Gcr_v }} \text{ KN} \\
    G_{tr,φ} &= G_{tr} \cdot φ_1  = {{ "%.3f" % QR2.Gtr_v }} \text{ KN} \\
    Qh       &= Q_{nom} \cdot φ_3 = {{ "%.3f" % QR2.Qh }}   \text{ KN}
\end{align*}

Αφόρτιστη γερανογέφυρα
\begin{align*}
    \sum{Q_{r,min}}   &= \dfrac{G_{cr,φ}}{2} + \dfrac{G_{tr,φ} \cdot (L - a)}{L} = {{ "%.3f" % QR2.SQr_max }} \text{ KN} \\
    Q_{r,min}         &= \dfrac{\displaystyle\sum{Q_{r,min}}}{2}                 = {{ "%.3f" % QR2.Qr_max }}  \text{ KN} \\
    \sum{Q_{r,(min)}} &= \dfrac{G_{cr,φ}}{2} + \dfrac{G_{tr,φ} \cdot a}{L}       = {{ "%.3f" % QR2.SQr_min }} \text{ KN} \\
    Q_{r,(min)}       &= \dfrac{\displaystyle\sum{Q_{r,(min)}}}{2}               = {{ "%.3f" % QR2.Qr_min }}  \text{ KN}
\end{align*}

Φορτισμένη γερανογέφυρα
\begin{align*}
    \sum{Q_{r,\max}}     &= \dfrac{G_{cr,φ}}{2} + \dfrac{(G_{tr,φ} + Qh) \cdot (L - a)}{L} = {{ "%.3f" % QR2.SQr_MAX }} \text{ KN} \\
    Q_{r,\max}           &= \dfrac{\displaystyle\sum{Q_{r,\max}}}{2}                       = {{ "%.3f" % QR2.Qr_MAX }}  \text{ KN} \\
    \sum{Q_{r,(\max)}}   &= \dfrac{G_{cr,φ}}{2} + \dfrac{(G_{tr,φ} + Qh) \cdot a}{L}       = {{ "%.3f" % QR2.SQr_MIN }} \text{ KN} \\
    Q_{r,(\max)}         &= \dfrac{\displaystyle\sum{Q_{r,(min)}}}{2}                      = {{ "%.3f" % QR2.Qr_MIN }} \text{ KN}
\end{align*}

\subsubsection{3ος συνδυασμός φορτίων}
Δυναμικά  φορτία
\begin{align*}
    G_{cr,φ} &= G_{cr} \cdot 1.00  = {{ "%.3f" % QR3.Gcr_v }} \text{ KN} \\
    G_{tr,φ} &= G_{tr} \cdot 1.00  = {{ "%.3f" % QR3.Gtr_v }} \text{ KN} \\
    Qh       &= Q_{nom} \cdot 0.00 = {{ "%.3f" % QR3.Qh }}   \text{ KN}
\end{align*}

Αφόρτιστη γερανογέφυρα
\begin{align*}
    \sum{Q_{r,min}}   &= \dfrac{G_{cr,φ}}{2} + \dfrac{G_{tr,φ} \cdot (L - a)}{L} = {{ "%.3f" % QR3.SQr_max }} \text{ KN} \\
    Q_{r,min}         &= \dfrac{\displaystyle\sum{Q_{r,min}}}{2}                 = {{ "%.3f" % QR3.Qr_max }}  \text{ KN} \\
    \sum{Q_{r,(min)}} &= \dfrac{G_{cr,φ}}{2} + \dfrac{G_{tr,φ} \cdot a}{L}       = {{ "%.3f" % QR3.SQr_min }} \text{ KN} \\
    Q_{r,(min)}       &= \dfrac{\displaystyle\sum{Q_{r,(min)}}}{2}               = {{ "%.3f" % QR3.Qr_min }}  \text{ KN}
\end{align*}

Φορτισμένη γερανογέφυρα
\begin{align*}
    \sum{Q_{r,\max}}     &= \dfrac{G_{cr,φ}}{2} + \dfrac{(G_{tr,φ} + Qh) \cdot (L - a)}{L} = {{ "%.3f" % QR3.SQr_MAX }} \text{ KN} \\
    Q_{r,\max}           &= \dfrac{\displaystyle\sum{Q_{r,\max}}}{2}                       = {{ "%.3f" % QR3.Qr_MAX }}  \text{ KN} \\
    \sum{Q_{r,(\max)}}   &= \dfrac{G_{cr,φ}}{2} + \dfrac{(G_{tr,φ} + Qh) \cdot a}{L}       = {{ "%.3f" % QR3.SQr_MIN }} \text{ KN} \\
    Q_{r,(\max)}         &= \dfrac{\displaystyle\sum{Q_{r,(min)}}}{2}                      = {{ "%.3f" % QR3.Qr_MIN }} \text{ KN}
\end{align*}

\subsubsection{4ος συνδυασμός φορτίων}

Δυναμικά  φορτία
\begin{align*}
    G_{cr,φ} &= G_{cr} \cdot φ_4  = {{ "%.3f" % QR4.Gcr_v }} \text{ KN} \\
    G_{tr,φ} &= G_{tr} \cdot φ_4  = {{ "%.3f" % QR4.Gtr_v }} \text{ KN} \\
    Qh       &= Q_{nom} \cdot φ_4 = {{ "%.3f" % QR4.Qh }}   \text{ KN}
\end{align*}

Αφόρτιστη γερανογέφυρα
\begin{align*}
    \sum{Q_{r,min}}   &= \dfrac{G_{cr,φ}}{2} + \dfrac{G_{tr,φ} \cdot (L - a)}{L} = {{ "%.3f" % QR4.SQr_max }} \text{ KN} \\
    Q_{r,min}         &= \dfrac{\displaystyle\sum{Q_{r,min}}}{2}                 = {{ "%.3f" % QR4.Qr_max }}  \text{ KN} \\
    \sum{Q_{r,(min)}} &= \dfrac{G_{cr,φ}}{2} + \dfrac{G_{tr,φ} \cdot a}{L}       = {{ "%.3f" % QR4.SQr_min }} \text{ KN} \\
    Q_{r,(min)}       &= \dfrac{\displaystyle\sum{Q_{r,(min)}}}{2}               = {{ "%.3f" % QR4.Qr_min }}  \text{ KN}
\end{align*}

Φορτισμένη γερανογέφυρα
\begin{align*}
    \sum{Q_{r,\max}}     &= \dfrac{G_{cr,φ}}{2} + \dfrac{(G_{tr,φ} + Qh) \cdot (L - a)}{L} = {{ "%.3f" % QR4.SQr_MAX }} \text{ KN} \\
    Q_{r,\max}           &= \dfrac{\displaystyle\sum{Q_{r,\max}}}{2}                       = {{ "%.3f" % QR4.Qr_MAX }}  \text{ KN} \\
    \sum{Q_{r,(\max)}}   &= \dfrac{G_{cr,φ}}{2} + \dfrac{(G_{tr,φ} + Qh) \cdot a}{L}       = {{ "%.3f" % QR4.SQr_MIN }} \text{ KN} \\
    Q_{r,(\max)}         &= \dfrac{\displaystyle\sum{Q_{r,(min)}}}{2}                      = {{ "%.3f" % QR4.Qr_MIN }} \text{ KN}
\end{align*}

\subsubsection{5oς συνδυασμός φορτίων}Ομοίως με τον 4ο.

\subsubsection{6oς συνδυασμός φορτίων}Ομοίως με τον 4ο.

\subsection{Οριζόντια φορτία λόγω επιτάχυνσης ή επιβράδυνσης}

Η κινητήρια δύναμη της γερανογέφυρας $K$ ασκείται στον άξονα της κίνησής της. H δύναμη αυτή
ισομοιράζεται σε δύο οριζόντιες κατά μήκος της τροχιάς δυνάμεις $H_L$ ασκούμενες στους τροχούς. Το
κέντρο βάρους όμως του συστήματος γερανογέφυρα - αναρτημένο φορτίο, βρίσκεται σε απόσταση $l_s$ από
τον άξονα της κίνησης, με αποτέλεσμα να παράγεται μια στρεπτική ροπή $M$. Για λόγους ισορροπίας
λοιπόν, αναπτύσσεται το ζεύγος των αντίρροπων δυνάμεων $H_{T1}$ - $H_{T2}$ στις θέσεις των τροχών.

\subsubsection{1ος και 2ος συνδυασμός φόρτισης}

Για τους δύο πρώτους συνδυασμούς φόρτισης ισχύει ότι:
\begin{align*}
    K         &= μ \cdot m_w \cdot Q_{r,(min)}  = {{ "%.3f" % HT1.K }} \text{KN} \\
    H_{L,1}   &= \dfrac{φ_5 \cdot K} {n_r} = {{ "%.3f" % HT1.HL }} \text{KN} \\
    H_{L,2}   &= \dfrac{φ_5 \cdot K} {n_r} = {{ "%.3f" % HT1.HL }} \text{KN} \\
    ξ_1       &= \dfrac{\displaystyle\sum{Q_{r,max}}}{\displaystyle\sum{Q_{r,max} + \displaystyle\sum{Q_{r,(max)}}}} = {{ "%.3f" % HT1.ksi1 }} \\
    ξ_2       &= 1 - ξ1 = {{ "%.3f" % HT1.ksi2 }} \\
    l_s       &= (ξ_1 - 0.5) \cdot ξ_2 = {{ "%.3f" % HT1.Ls }} \text{m} \\
    M         &= K \cdot l_s = {{ "%.3f" % HT1.M }} \text{KNm} \\
    H_{T1}    &= φ_5 \cdot ξ_2 \cdot \dfrac{M}{a} = {{ "%.3f" % HT1.HT_1 }} \text{KN} \\
    H_{T2}    &= φ_5 \cdot ξ_1 \cdot \dfrac{M}{a} = {{ "%.3f" % HT1.HT_2 }} \text{KN}
\end{align*}

\subsubsection{3ος, 4ος και 8ος συνδυασμός φόρτισης}

Για τον 3ο, τον 4ο και τον 8ο συνδυασμό φόρτισης ισχύει ότι:
\begin{align*}
    K         &= μ \cdot m_w \cdot Q_{r,(min)}  = {{ "%.3f" % HT4.K }} \text{KN} \\
    H_{L,1}   &= \dfrac{φ_5 \cdot K} {n_r} = {{ "%.3f" % HT4.HL }} \text{KN} \\
    H_{L,2}   &= \dfrac{φ_5 \cdot K} {n_r} = {{ "%.3f" % HT4.HL }} \text{KN} \\
    ξ_1       &= \dfrac{\displaystyle\sum{Q_{r,max}}}{\displaystyle\sum{Q_{r,max} + \displaystyle\sum{Q_{r,(max)}}}} = {{ "%.3f" % HT4.ksi1 }} \\
    ξ_2       &= 1 - ξ1 = {{ "%.3f" % HT4.ksi2 }} \\
    l_s       &= (ξ_1 - 0.5) \cdot ξ_2 = {{ "%.3f" % HT4.Ls }} \text{m} \\
    M         &= K \cdot l_s = {{ "%.3f" % HT4.M }} \text{KNm} \\
    H_{T1}    &= φ_5 \cdot ξ_2 \cdot \dfrac{M}{a_r} = {{ "%.3f" % HT4.HT_1 }} \text{KN} \\
    H_{T2}    &= φ_5 \cdot ξ_1 \cdot \dfrac{M}{a_r} = {{ "%.3f" % HT4.HT_2 }} \text{KN}
\end{align*}
