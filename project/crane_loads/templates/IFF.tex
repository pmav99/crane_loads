
\subsection{Οριζόντια φορτία λόγω παράγωγης κίνησης της γερανογέφυρας (συνδυασμός 5)}
Κατά την κίνηση της η γερανογέφυρα ακολουθεί μια λοξή πορεία ως προς τις τροχιές της. Το γεγονός αυτό έχει ως συνέπεια τη δημιουργία των εξής δυνάμεων:
\begin{itemize} 
\item της οδηγούσας δύναμης $S$ που ασκείται στη θέση των μέσων καθοδήγησης, και
\item των εγκάρσιων ως προς τις τροχιές δυνάμεων που ασκούνται σε κάθε τροχό.
\end{itemize}

\begin{align*}
\sum{Q_r} &= \displaystyle\sum{Q_{r,max}} + \displaystyle\sum{Q_{r,(max)}} = SQr-- \text{ KN} \\ 
ξ_1    	  &= \dfrac{\displaystyle\sum{Q_{r,max}}}{\displaystyle\sum{Q_{r,max}} + \displaystyle\sum{Q_{r,(max)}}} = ksi1-- \text{ KN} \\ 
ξ_2       &= 1 - ξ1 = ksi2-- \text{ KN} \\ 
α    	  &= a_rad-- \text{ rad} \\ 
f         &= 0.3 \times \left(1 - e^{-250 \times α}\right) = f--
\end{align*}

Αφού το σύστημα κύλισης της γερανογέφυρας είναι IFF έχουμε:
\begin{align*}
h          &= \dfrac{m \times ξ_1 \times ξ_2 \times L^2 + \sum{e_i^2}}{\sum{e_i}} = h-- \\ 
λ_s        &= 1 - \dfrac{\sum{e_i}}{n \times h} =  l_s-- \\ 
λ_{S,11,L} &= l_s11L-- \\ 
λ_{S,12,L} &= l_s12L-- \\ 
λ_{S,21,L} &= l_s21L-- \\ 
λ_{S,22,L} &= l_s22L-- \\ 
λ_{S,11,T} &= \dfrac{ξ_2}{n} \times \left(1 - \dfrac{e_1}{h}\right) = l_s11T-- \\ 
λ_{S,12,T} &= \dfrac{ξ_2}{n} \times \left(1 - \dfrac{e_2}{h}\right) = l_s12T-- \\ 
λ_{S,21,T} &= \dfrac{ξ_1}{n} \times \left(1 - \dfrac{e_1}{h}\right) = l_s21T-- \\ 
λ_{S,22,T} &= \dfrac{ξ_1}{n} \times \left(1 - \dfrac{e_2}{h}\right) = l_s22T--
\end{align*}

Άρα οι ασκούμενες δυνάμεις είναι:
\begin{align*}
S                &= f \times λ_s \times \sum{Q_r} = S-- \text{ KN} \\ 
H_{S,11,L}       &= f \times λ_{s,11,L} \times \sum{Q_r} = H_s11L-- \text{ KN} \\ 
H_{S,12,L}       &= f \times λ_{s,12,L} \times \sum{Q_r} = H_s12L-- \text{ KN} \\ 
H_{S,21,L}       &= f \times λ_{s,21,L} \times \sum{Q_r} = H_s21L-- \text{ KN} \\ 
H_{S,22,L}       &= f \times λ_{s,22,L} \times \sum{Q_r} = H_s22L-- \text{ KN} \\ 
H_{S,11,T}       &= f \times λ_{s,11,T} \times \sum{Q_r} = H_s11T-- \text{ KN} \\ 
H_{S,12,T}       &= f \times λ_{s,12,T} \times \sum{Q_r} = H_s12T-- \text{ KN} \\ 
H_{S,21,T}       &= f \times λ_{s,21,T} \times \sum{Q_r} = H_s21T-- \text{ KN} \\ 
H_{S,22,T}       &= f \times λ_{s,22,T} \times \sum{Q_r} = H_s22T-- \text{ KN} 
\end{align*}

Τελικά στη θέση του προπορευόμενου ζεύγους τροχών έχουμε:
\begin{align*}
H_{S,1T} &= H_{S,11,T} - S = H_s1T-- \text{ KN} \\
H_{S,2T} &= H_{S,21,T} = H_s21T-- \text{ KN}
\end{align*}
