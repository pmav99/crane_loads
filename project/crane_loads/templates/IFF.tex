
\subsection{Οριζόντια φορτία λόγω παράγωγης κίνησης της γερανογέφυρας (συνδυασμός 5)}

Κατά την κίνηση της η γερανογέφυρα ακολουθεί μια λοξή πορεία ως προς τις τροχιές της. Το γεγονός
αυτό έχει ως συνέπεια τη δημιουργία των εξής δυνάμεων:

\begin{itemize}
    \item της οδηγούσας δύναμης $S$ που ασκείται στη θέση των μέσων καθοδήγησης, και
    \item των εγκάρσιων ως προς τις τροχιές δυνάμεων που ασκούνται σε κάθε τροχό.
\end{itemize}

\begin{align*}
    \sum{Q_r}   &= \displaystyle\sum{Q_{r,max}} + \displaystyle\sum{Q_{r,(max)}} = {{ SQr }} \text{ KN} \\
    ξ_1         &= \dfrac{\displaystyle\sum{Q_{r,max}}}{\displaystyle\sum{Q_{r,max}} + \displaystyle\sum{Q_{r,(max)}}} = {{ ksi1 }} \text{ KN} \\
    ξ_2         &= 1 - ξ1 = {{ ksi2 }} \text{ KN} \\
    α           &= {{ a_rad }} \text{ rad} \\
    f           &= 0.3 \cdot \left(1 - e^{-250 \cdot α}\right) = {{ f }}
\end{align*}

Αφού το σύστημα κύλισης της γερανογέφυρας είναι IFF έχουμε:
\begin{align*}
    h          &= \dfrac{m \cdot ξ_1 \cdot ξ_2 \cdot L^2 + \sum{e_i^2}}{\sum{e_i}} = {{ h }} \\
    λ_s        &= 1 - \dfrac{\sum{e_i}}{n \cdot h} =  {{ l_s }} \\
    λ_{S,11,L} &= {{ l_s11L }} \\
    λ_{S,12,L} &= {{ l_s12L }} \\
    λ_{S,21,L} &= {{ l_s21L }} \\
    λ_{S,22,L} &= {{ l_s22L }} \\
    λ_{S,11,T} &= \dfrac{ξ_2}{n} \cdot \left(1 - \dfrac{e_1}{h}\right) = {{ l_s11T }} \\
    λ_{S,12,T} &= \dfrac{ξ_2}{n} \cdot \left(1 - \dfrac{e_2}{h}\right) = {{ l_s12T }} \\
    λ_{S,21,T} &= \dfrac{ξ_1}{n} \cdot \left(1 - \dfrac{e_1}{h}\right) = {{ l_s21T }} \\
    λ_{S,22,T} &= \dfrac{ξ_1}{n} \cdot \left(1 - \dfrac{e_2}{h}\right) = {{ l_s22T }}
\end{align*}

Άρα οι ασκούμενες δυνάμεις είναι:
\begin{align*}
    S                &= f \cdot λ_s \cdot \sum{Q_r} = {{ S }} \text{ KN} \\
    H_{S,11,L}       &= f \cdot λ_{s,11,L} \cdot \sum{Q_r} = {{ H_s11L }} \text{ KN} \\
    H_{S,12,L}       &= f \cdot λ_{s,12,L} \cdot \sum{Q_r} = {{ H_s12L }} \text{ KN} \\
    H_{S,21,L}       &= f \cdot λ_{s,21,L} \cdot \sum{Q_r} = {{ H_s21L }} \text{ KN} \\
    H_{S,22,L}       &= f \cdot λ_{s,22,L} \cdot \sum{Q_r} = {{ H_s22L }} \text{ KN} \\
    H_{S,11,T}       &= f \cdot λ_{s,11,T} \cdot \sum{Q_r} = {{ H_s11T }} \text{ KN} \\
    H_{S,12,T}       &= f \cdot λ_{s,12,T} \cdot \sum{Q_r} = {{ H_s12T }} \text{ KN} \\
    H_{S,21,T}       &= f \cdot λ_{s,21,T} \cdot \sum{Q_r} = {{ H_s21T }} \text{ KN} \\
    H_{S,22,T}       &= f \cdot λ_{s,22,T} \cdot \sum{Q_r} = {{ H_s22T }} \text{ KN}
\end{align*}

Τελικά στη θέση του προπορευόμενου ζεύγους τροχών έχουμε:
\begin{align*}
    H_{S,1T} &= H_{S,11,T} - S = {{ H_s1T }} \text{ KN} \\
    H_{S,2T} &= H_{S,21,T} = {{ H_s21T }} \text{ KN}
\end{align*}
