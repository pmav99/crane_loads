\section{Δεδομένα}
\subsection{Γενικά στοιχεία}
Οι συντελεστές $φ_{2,min}$ και $β_2$ εξαρτώνται από την κατηγορία ανύψωσης της γερανογέφυρας και υπολογίζονται βάσει του πίνακα~2.5 του EC1991-3:2006.

\begin{tabular}{llcr}
    Κατηγορία κόπωσης                   &$=$ {{ FC }} \\
    Σύστημα κύλισης                     &$=$ {{ RT }}\\
    Τροχοί ανά άξονα    $n_r$           &$=$ ${{ nr }} $ \\
    Μεμονωμένοι κινητήριοι τροχοί $m_w$ &$=$ ${{ mw }}$ \\
    Συντελεστής τριβής $\mu$            &$=$ ${{ mf }}$ \\
    Κατηγορία ανύψωσης                  &$=$ {{ HC }} \\
    $φ_{2,min}$                         &$=$ ${{ v2_min }}$ \\
    $β_2$                               &$=$ ${{ b2 }}$ \\
    Ταχύτητα ανύψωσης  $V_h$            &$=$ ${{ vh }}$ m/sec
\end{tabular}

\subsection{Φορτία}
Οι τιμές των φορτίων είναι χαρακτηριστικές τιμές.

\begin{tabular}{llcr}
    Ίδιο βάρος γερανογέφυρας     & $G_{cr}$  &$=$ &${{ Gcr }}$ KN \\
    Ίδιο βάρος φορείου           & $G_{tr}$  &$=$ &${{ Gtr }}$ KN \\
    Συνολικό ίδιο βάρος          & $G_{tot}$ &$=$ &${{ Gtot }}$ KN \\
    Ωφέλιμο φορτίο               & $Q_{nom}$ &$=$ &${{ Qr_nom }}$ KN
\end{tabular}

\subsection{Γεωμετρία}
Τα γεωμετρικά χαρακτηριστικά της γερανογέφυρας είναι:

\begin{tabular}{llcl}
    Άνοιγμα γερανογέφυρας                         &$L$          &$=$ &${{ L }}$ m \\
    Ελάχιστη απόσταση άγκιστρου από τις τροχιές   &$e_{min}$    &$=$ &${{ e_min }}$ m \\
    Απόσταση τροχών επί της δοκού κύλισης         &$a$          &$=$ &${{ a }}$ m \\
    Απόσταση τροχών $1$ από τα μέσα καθοδήγησης   &$e_1$        &$=$ &${{ e1 }}$ m \\
    Απόσταση τροχών $2$ από τα μέσα καθοδήγησης   &$e_2$        &$=$ &${{ e2 }}$ m \\
    Γωνία λοξότητας (skewing angle)               &$α$          &$=$ &${{ a_rad }}$ rad
\end{tabular}

\section{Δυναμικοί συντελεστές}
Οι δυναμικοί συντελεστές υπολογίζονται σύμφωνα με το ΕΝ1991-3:2006.

Ο συντελεστής $φ_1$ εφαρμόζεται στα μόνιμα φορτία και αντιπροσωπεύει τις ταλαντώσεις της
γερανογέφυρας. Ο συντελεστής $φ_2$ εφαρμόζεται στο ανυψούμενο φορτίο και η τιμή του διαφοροποιείται
ανάλογα με την κατηγορία ανύψωσης και την ταχύτητα ανύψωσης. Ο συντελεστής $φ_3$ εξαρτάται από τον
τρόπο απόθεσης του ανυψούμενου φορτίου. Υπέρ της ασφαλείας μπορεί να θεωρηθεί ίσος με τη μονάδα.
Ο συντελεστής $φ_4$, θεωρώντας ότι έχουν τηρηθεί οι ανοχές για τη σιδηροτροχιά που ορίζονται στο
ΕΝ1993-6:2007, μπορεί να λαμβάνεται και αυτός ίσος με τη μονάδα. Ο συντελεστής $φ_5$ εξαρτάται από
τον τρόπο με τον οποίο μεταβάλλονται οι δυνάμεις στη γερανογέφυρα. Ο συντελεστής $φ_6$ εξαρτάται από
το τρόπο διεξαγωγής της δοκιμαστικής φόρτισης καθώς και από το μέγεθος του δοκιμαστικού φορτίου.

\begin{tabular}{llcr}
    $φ_1$ &$= {{ v1 }} $ \\
    $φ_2$ &$= φ_{2,min} + β_2 \times V_h = {{ v2_min }} + {{ b2 }} \times {{ vh }} = {{ v2 }} $ \\
    $φ_3$ &$= {{ v3 }} $ \\
    $φ_4$ &$= {{ v4 }} $ \\
    $φ_5$ &$= {{ v5 }} $ \\
    $φ_6$ &$= {{ v6 }} $
\end{tabular}
