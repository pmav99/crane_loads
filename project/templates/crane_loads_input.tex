\section{Δεδομένα}
\subsection{Γενικά στοιχεία}
Οι συντελεστές $φ_{2,min}$ και $β_2$ εξαρτώνται από την κατηγορία ανύψωσης της γερανογέφυρας και υπολογίζονται βάσει του πίνακα~2.5 του EC1991-3:2006.

\begin{tabular}{llcr}
Κατηγορία κόπωσης					&$=$ S2 \\ 
Σύστημα κύλισης						&$=$ IFF \\ 
Τροχοί ανά άξονα	$n_r$			&$=$ $2$ \\ 
Μεμονωμένοι κινητήριοι τροχοί $m_w$ &$=$ $2$ \\
Συντελεστής τριβής $\mu$			&$=$ $0.200$ \\ 
Κατηγορία ανύψωσης					&$=$ HC2 \\
$φ_{2,min}$							&$=$ $1.100$ \\ 
$β_2$								&$=$ $0.340$ \\
Ταχύτητα ανύψωσης  $V_h$			&$=$ $0.333$ m/sec
\end{tabular}

\subsection{Φορτία}
Οι τιμές των φορτίων είναι χαρακτηριστικές τιμές.

\begin{tabular}{llcr}
Ίδιο βάρος γερανογέφυρας     & $G_{cr}$  &$=$ &$27.270$ KN \\ 
Ίδιο βάρος φορείου           & $G_{tr}$  &$=$ &$ 2.800$ KN \\ 
Συνολικό ίδιο βάρος          & $G_{tot}$ &$=$ &$30.070$ KN \\ 
Ωφέλιμο φορτίο               & $Q_{nom}$ &$=$ &$32.000$ KN 
\end{tabular}

\subsection{Γεωμετρία}
Τα γεωμετρικά χαρακτηριστικά της γερανογέφυρας είναι:

\begin{tabular}{llcl}
Άνοιγμα γερανογέφυρας                                 &$L$		  &$=$ &$15.000$ m \\ 
Ελάχιστη απόσταση άγκιστρου από τις τροχιές   		  &$e_{min}$  &$=$ &$0.780$ m \\ 
Απόσταση τροχών επί της δοκού κύλισης                 &$a$		  &$=$ &$1.600$ m \\ 
Απόσταση τροχών $1$ από τα μέσα καθοδήγησης           &$e_1$      &$=$ &$0.000$ m \\ 
Απόσταση τροχών $2$ από τα μέσα καθοδήγησης           &$e_2$      &$=$ &$1.600$ m \\
Γωνία λοξότητας (skewing angle)						  &$α$		  &$=$ &$0.0015$ rad
\end{tabular}

\section{Δυναμικοί συντελεστές}
Οι δυναμικοί συντελεστές υπολογίζονται σύμφωνα με το ΕΝ1991-3:2006.

Ο συντελεστής $φ_1$ εφαρμόζεται στα μόνιμα φορτία και αντιπροσωπεύει τις ταλαντώσεις της γερανογέφυρας. Ο συντελεστής $φ_2$ εφαρμόζεται στο ανυψούμενο φορτίο και η τιμή του διαφοροποιείται ανάλογα με την κατηγορία ανύψωσης και την ταχύτητα ανύψωσης. Ο συντελεστής $φ_3$ εξαρτάται από τον τρόπο απόθεσης του ανυψούμενου φορτίου. Υπέρ της ασφαλείας μπορεί να θεωρηθεί ίσος με τη μονάδα. Ο συντελεστής $φ_4$, θεωρώντας ότι έχουν τηρηθεί οι ανοχές για τη σιδηροτροχιά που ορίζονται στο ΕΝ1993-6:2007, μπορεί να λαμβάνεται και αυτός ίσος με τη μονάδα. Ο συντελεστής $φ_5$ εξαρτάται από τον τρόπο με τον οποίο μεταβάλλονται οι δυνάμεις στη γερανογέφυρα. Ο συντελεστής $φ_6$ εξαρτάται από το τρόπο διεξαγωγής της δοκιμαστικής φόρτισης καθώς και από το μέγεθος του δοκιμαστικού φορτίου.

\begin{tabular}{llcr}
$φ_1$ &$= 1.100 $ \\ 
$φ_2$ &$= φ_{2,min} + β_2 \times V_h = 1.100 + 0.340 \times 0.333 = 1.213 $ \\ 
$φ_3$ &$= 1.000 $ \\ 
$φ_4$ &$= 1.000 $ \\ 
$φ_5$ &$= 1.500 $ \\ 
$φ_6$ &$= 1.000 $ 
\end{tabular}

