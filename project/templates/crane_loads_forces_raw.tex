\section{Φορτία}
Συνολικά ορίζονται 10 συνδυασμοί φόρτισης. Ο υπολογισμός των φορτίων γίνεται πολλαπλασιάζοντας τις χαρακτηριστικές τιμές των των ιδίων βαρών και των ωφέλιμων φορτίων με τους αντίστοιχους δυναμικούς συντελεστές $φ_i$, όπως ορίζονται στον πίνακα~2.2 του EC1991-3:2006. Στη συνέχεια, και για κάθε συνδυασμό φόρτισης, υπολογίζονται τα μέγιστα (φορτισμένη γερανογέφυρα) και τα ελάχιστα (αφόρτιστη γερανογέφυρα) φορτία τροχού της γερανογέφυρας.

\subsection{Κατακόρυφα φορτία}
\subsubsection{1ος συνδυασμός φορτίων}
Δυναμικά  φορτία
\begin{align*}
G_{cr,φ}  &= G_{cr} \times φ_1  = Gcr_v-- \text{ KN} \\
G_{tr,φ}  &= G_{tr} \times φ_1  = Gtr_v-- \text{ KN} \\
Q_h       &= Q_{nom} \times φ_2 = Q_h--   \text{ KN}
\end{align*}

Αφόρτιστη γερανογέφυρα
\begin{align*}
\sum{Q_{r,min}}   &= \dfrac{G_{cr,φ}}{2} + \dfrac{G_{tr,φ} \times (L - a)}{L} = SQr_max-- \text{ KN} \\
Q_{r,min}         &= \dfrac{\displaystyle\sum{Q_{r,min}}}{2}                  = Qr_max--  \text{ KN} \\
\sum{Q_{r,(min)}} &= \dfrac{G_{cr,φ}}{2} + \dfrac{G_{tr,φ} \times a}{L}       = SQr_min-- \text{ KN} \\
Q_{r,(min)}       &= \dfrac{\displaystyle\sum{Q_{r,(min)}}}{2}                = Qr_min--  \text{ KN}
\end{align*}

Φορτισμένη γερανογέφυρα
\begin{align*}
\sum{Q_{r,\max}}     &= \dfrac{G_{cr,φ}}{2} + \dfrac{(G_{tr,φ} + Q_h) \times (L - a)}{L}   = SQr_MAX-- \text{ KN} \\
Q_{r,\max}           &= \dfrac{\displaystyle\sum{Q_{r,\max}}}{2}                           = Qr_MAX--  \text{ KN} \\
\sum{Q_{r,(\max)}}   &= \dfrac{G_{cr,φ}}{2} + \dfrac{(G_{tr,φ} + Q_h) \times a}{L}         = SQr_MIN-- \text{ KN} \\
Q_{r,(\max)}         &= \dfrac{\displaystyle\sum{Q_{r,(min)}}}{2}                          = Qr_MIN-- \text{ KN}
\end{align*}

\subsubsection{2ος συνδυασμός φορτίων}
Δυναμικά  φορτία
\begin{align*}
G_{cr,φ}  &= G_{cr} \times φ_1    = Gcr_v-- \text{ KN} \\
G_{tr,φ}  &= G_{tr} \times φ_1    = Gtr_v-- \text{ KN} \\
Q_h       &= Q_{nom} \times φ_3   = Q_h--   \text{ KN}
\end{align*}

Αφόρτιστη γερανογέφυρα
\begin{align*}
\sum{Q_{r,min}}   &= \dfrac{G_{cr,φ}}{2} + \dfrac{G_{tr,φ} \times (L - a)}{L} = SQr_max-- \text{ KN} \\
Q_{r,min}         &= \dfrac{\displaystyle\sum{Q_{r,min}}}{2}                  = Qr_max--  \text{ KN} \\
\sum{Q_{r,(min)}} &= \dfrac{G_{cr,φ}}{2} + \dfrac{G_{tr,φ} \times a}{L}       = SQr_min-- \text{ KN} \\
Q_{r,(min)}       &= \dfrac{\displaystyle\sum{Q_{r,(min)}}}{2}                = Qr_min--  \text{ KN}
\end{align*}

Φορτισμένη γερανογέφυρα
\begin{align*}
\sum{Q_{r,\max}}     &= \dfrac{G_{cr,φ}}{2} + \dfrac{(G_{tr,φ} + Q_h) \times (L - a)}{L}   = SQr_MAX-- \text{ KN} \\
Q_{r,\max}           &= \dfrac{\displaystyle\sum{Q_{r,\max}}}{2}                           = Qr_MAX--  \text{ KN} \\
\sum{Q_{r,(\max)}}   &= \dfrac{G_{cr,φ}}{2} + \dfrac{(G_{tr,φ} + Q_h) \times a}{L}         = SQr_MIN-- \text{ KN} \\
Q_{r,(\max)}         &= \dfrac{\displaystyle\sum{Q_{r,(min)}}}{2}                          = Qr_MIN-- \text{ KN}
\end{align*}

\subsubsection{3ος συνδυασμός φορτίων}
Δυναμικά  φορτία
\begin{align*}
G_{cr,φ}  &= G_{cr} \times 1.00    = Gcr_v-- \text{ KN} \\
G_{tr,φ}  &= G_{tr} \times 1.00    = Gtr_v-- \text{ KN} \\
Q_h       &= Q_{nom} \times 0.00   = Q_h--   \text{ KN}
\end{align*}

Αφόρτιστη γερανογέφυρα
\begin{align*}
\sum{Q_{r,min}}   &= \dfrac{G_{cr,φ}}{2} + \dfrac{G_{tr,φ} \times (L - a)}{L} = SQr_max-- \text{ KN} \\
Q_{r,min}         &= \dfrac{\displaystyle\sum{Q_{r,min}}}{2}                  = Qr_max--  \text{ KN} \\
\sum{Q_{r,(min)}} &= \dfrac{G_{cr,φ}}{2} + \dfrac{G_{tr,φ} \times a}{L}       = SQr_min-- \text{ KN} \\
Q_{r,(min)}       &= \dfrac{\displaystyle\sum{Q_{r,(min)}}}{2}                = Qr_min--  \text{ KN}
\end{align*}

Φορτισμένη γερανογέφυρα
\begin{align*}
\sum{Q_{r,\max}}     &= \dfrac{G_{cr,φ}}{2} + \dfrac{(G_{tr,φ} + Q_h) \times (L - a)}{L}   = SQr_MAX-- \text{ KN} \\
Q_{r,\max}           &= \dfrac{\displaystyle\sum{Q_{r,\max}}}{2}                           = Qr_MAX--  \text{ KN} \\
\sum{Q_{r,(\max)}}   &= \dfrac{G_{cr,φ}}{2} + \dfrac{(G_{tr,φ} + Q_h) \times a}{L}         = SQr_MIN-- \text{ KN} \\
Q_{r,(\max)}         &= \dfrac{\displaystyle\sum{Q_{r,(min)}}}{2}                          = Qr_MIN-- \text{ KN}
\end{align*}

\subsubsection{4ος συνδυασμός φορτίων}
Δυναμικά  φορτία
\begin{align*}
G_{cr,φ}  &= G_{cr} \times φ_4    = Gcr_v-- \text{ KN} \\
G_{tr,φ}  &= G_{tr} \times φ_4    = Gtr_v-- \text{ KN} \\
Q_h       &= Q_{nom} \times φ_4   = Q_h--   \text{ KN}
\end{align*}

Αφόρτιστη γερανογέφυρα
\begin{align*}
\sum{Q_{r,min}}   &= \dfrac{G_{cr,φ}}{2} + \dfrac{G_{tr,φ} \times (L - a)}{L} = SQr_max-- \text{ KN} \\
Q_{r,min}         &= \dfrac{\displaystyle\sum{Q_{r,min}}}{2}                  = Qr_max--  \text{ KN} \\
\sum{Q_{r,(min)}} &= \dfrac{G_{cr,φ}}{2} + \dfrac{G_{tr,φ} \times a}{L}       = SQr_min-- \text{ KN} \\
Q_{r,(min)}       &= \dfrac{\displaystyle\sum{Q_{r,(min)}}}{2}                = Qr_min--  \text{ KN}
\end{align*}

Φορτισμένη γερανογέφυρα
\begin{align*}
\sum{Q_{r,\max}}     &= \dfrac{G_{cr,φ}}{2} + \dfrac{(G_{tr,φ} + Q_h) \times (L - a)}{L}   = SQr_MAX-- \text{ KN} \\
Q_{r,\max}           &= \dfrac{\displaystyle\sum{Q_{r,\max}}}{2}                           = Qr_MAX--  \text{ KN} \\
\sum{Q_{r,(\max)}}   &= \dfrac{G_{cr,φ}}{2} + \dfrac{(G_{tr,φ} + Q_h) \times a}{L}         = SQr_MIN-- \text{ KN} \\
Q_{r,(\max)}         &= \dfrac{\displaystyle\sum{Q_{r,(min)}}}{2}                          = Qr_MIN-- \text{ KN}
\end{align*}

\subsubsection{5oς συνδυασμός φορτίων}Ομοίως με τον 4ο.

\subsubsection{6oς συνδυασμός φορτίων}Ομοίως με τον 4ο.

\subsection{Οριζόντια φορτία λόγω επιτάχυνσης ή επιβράδυνσης}
Η κινητήρια δύναμη της γερανογέφυρας $K$ ασκείται στον άξονα της κίνησής της. H δύναμη αυτή ισομοιράζεται σε δύο οριζόντιες κατά μήκος της τροχιάς δυνάμεις $H_L$ ασκούμενες στους τροχούς. Το κέντρο βάρους όμως του συστήματος γερανογέφυρα - αναρτημένο φορτίο, βρίσκεται σε απόσταση $l_s$ από τον άξονα της κίνησης, με αποτέλεσμα να παράγεται μια στρεπτική ροπή $M$. Για λόγους ισορροπίας λοιπόν, αναπτύσσεται το ζεύγος των αντίρροπων δυνάμεων $H_{T1}$ - $H_{T2}$ στις θέσεις των τροχών.

\subsubsection{1ος και 2ος συνδυασμός φόρτισης}
Για τους δύο πρώτους συνδυασμούς φόρτισης ισχύει ότι:
\begin{align*}
K         &= μ \times m_w \times Q_{r,(min)}  = K-- \text{KN} \\ 
H_{L,1}   &= \dfrac{φ_5 \times K} {n_r} = HL-- \text{KN} \\ 
H_{L,2}   &= \dfrac{φ_5 \times K} {n_r} = HL-- \text{KN} \\ 
ξ_1       &= \dfrac{\displaystyle\sum{Q_{r,max}}}{\displaystyle\sum{Q_{r,max} + \displaystyle\sum{Q_{r,(max)}}}} = ksi1-- \text{KN} \\ 
ξ_2       &= 1 - ξ1 = ksi2-- \text{KN} \\ 
l_s       &= (ξ_1 - 0.5) \times ξ_2 = Ls-- \text{KN} \\ 
M         &= K \times l_s = M-- \text{KN} \\ 
H_{T1}    &= φ_5 \times ξ_2 \times \dfrac{M}{a} = HT_1-- \text{KN} \\ 
H_{T2}    &= φ_5 \times ξ_1 \times \dfrac{M}{a} = HT_2-- \text{KN} 
\end{align*}

\subsubsection{3ος, 4ος και 8ος συνδυασμός φόρτισης}
Για τον 3ο, τον 4ο και τον 8ο συνδυασμό φόρτισης ισχύει ότι:
\begin{align*}
K         &= μ \times m_w \times Q_{r,(min)}  = K-- \text{KN} \\ 
H_{L,1}   &= \dfrac{φ_5 \times K} {n_r} = HL-- \text{KN} \\ 
H_{L,2}   &= \dfrac{φ_5 \times K} {n_r} = HL-- \text{KN} \\ 
ξ_1       &= \dfrac{\displaystyle\sum{Q_{r,max}}}{\displaystyle\sum{Q_{r,max} + \displaystyle\sum{Q_{r,(max)}}}} = ksi1-- \text{KN} \\ 
ξ_2       &= 1 - ξ1 = ksi2-- \text{KN} \\ 
l_s       &= (ξ_1 - 0.5) \times ξ_2 = Ls-- \text{KN} \\ 
M         &= K \times l_s = M-- \text{KN} \\ 
H_{T1}    &= φ_5 \times ξ_2 \times \dfrac{M}{a_r} = HT_1-- \text{KN} \\ 
H_{T2}    &= φ_5 \times ξ_1 \times \dfrac{M}{a_r} = HT_2-- \text{KN} 
\end{align*}



