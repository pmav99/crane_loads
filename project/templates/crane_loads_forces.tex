\section{Φορτία}
Συνολικά ορίζονται 10 συνδυασμοί φόρτισης. Ο υπολογισμός των φορτίων γίνεται πολλαπλασιάζοντας τις χαρακτηριστικές τιμές των των ιδίων βαρών και των ωφέλιμων φορτίων με τους αντίστοιχους δυναμικούς συντελεστές $φ_i$, όπως ορίζονται στον πίνακα~2.2 του EC1991-3:2006. Στη συνέχεια, και για κάθε συνδυασμό φόρτισης, υπολογίζονται τα μέγιστα (φορτισμένη γερανογέφυρα) και τα ελάχιστα (αφόρτιστη γερανογέφυρα) φορτία τροχού της γερανογέφυρας.

\subsection{Κατακόρυφα φορτία}
\subsubsection{1ος συνδυασμός φορτίων}
Δυναμικά  φορτία
\begin{align*}
G_{cr,φ}  &= G_{cr} \times φ_1  =  30.00 \text{ KN} \\
G_{tr,φ}  &= G_{tr} \times φ_1  =   3.08 \text{ KN} \\
Q_h       &= Q_{nom} \times φ_2 =  38.83   \text{ KN}
\end{align*}

Αφόρτιστη γερανογέφυρα
\begin{align*}
\sum{Q_{r,min}}   &= \dfrac{G_{cr,φ}}{2} + \dfrac{G_{tr,φ} \times (L - a)}{L} =  17.92 \text{ KN} \\
Q_{r,min}         &= \dfrac{\displaystyle\sum{Q_{r,min}}}{2}                  =   8.96  \text{ KN} \\
\sum{Q_{r,(min)}} &= \dfrac{G_{cr,φ}}{2} + \dfrac{G_{tr,φ} \times a}{L}       =  15.16 \text{ KN} \\
Q_{r,(min)}       &= \dfrac{\displaystyle\sum{Q_{r,(min)}}}{2}                =   7.58  \text{ KN}
\end{align*}

Φορτισμένη γερανογέφυρα
\begin{align*}
\sum{Q_{r,\max}}     &= \dfrac{G_{cr,φ}}{2} + \dfrac{(G_{tr,φ} + Q_h) \times (L - a)}{L}   =  54.73 \text{ KN} \\
Q_{r,\max}           &= \dfrac{\displaystyle\sum{Q_{r,\max}}}{2}                           =  27.36  \text{ KN} \\
\sum{Q_{r,(\max)}}   &= \dfrac{G_{cr,φ}}{2} + \dfrac{(G_{tr,φ} + Q_h) \times a}{L}         =  17.18 \text{ KN} \\
Q_{r,(\max)}         &= \dfrac{\displaystyle\sum{Q_{r,(min)}}}{2}                          =   8.59 \text{ KN}
\end{align*}

\subsubsection{2ος συνδυασμός φορτίων}
Δυναμικά  φορτία
\begin{align*}
G_{cr,φ}  &= G_{cr} \times φ_1    =  30.00 \text{ KN} \\
G_{tr,φ}  &= G_{tr} \times φ_1    =   3.08 \text{ KN} \\
Q_h       &= Q_{nom} \times φ_3   =  32.00   \text{ KN}
\end{align*}

Αφόρτιστη γερανογέφυρα
\begin{align*}
\sum{Q_{r,min}}   &= \dfrac{G_{cr,φ}}{2} + \dfrac{G_{tr,φ} \times (L - a)}{L} =  17.92 \text{ KN} \\
Q_{r,min}         &= \dfrac{\displaystyle\sum{Q_{r,min}}}{2}                  =   8.96  \text{ KN} \\
\sum{Q_{r,(min)}} &= \dfrac{G_{cr,φ}}{2} + \dfrac{G_{tr,φ} \times a}{L}       =  15.16 \text{ KN} \\
Q_{r,(min)}       &= \dfrac{\displaystyle\sum{Q_{r,(min)}}}{2}                =   7.58  \text{ KN}
\end{align*}

Φορτισμένη γερανογέφυρα
\begin{align*}
\sum{Q_{r,\max}}     &= \dfrac{G_{cr,φ}}{2} + \dfrac{(G_{tr,φ} + Q_h) \times (L - a)}{L}   =  48.25 \text{ KN} \\
Q_{r,\max}           &= \dfrac{\displaystyle\sum{Q_{r,\max}}}{2}                           =  24.13  \text{ KN} \\
\sum{Q_{r,(\max)}}   &= \dfrac{G_{cr,φ}}{2} + \dfrac{(G_{tr,φ} + Q_h) \times a}{L}         =  16.82 \text{ KN} \\
Q_{r,(\max)}         &= \dfrac{\displaystyle\sum{Q_{r,(min)}}}{2}                          =   8.41 \text{ KN}
\end{align*}

\subsubsection{3ος συνδυασμός φορτίων}
Δυναμικά  φορτία
\begin{align*}
G_{cr,φ}  &= G_{cr} \times 1.00    =  27.27 \text{ KN} \\
G_{tr,φ}  &= G_{tr} \times 1.00    =   2.80 \text{ KN} \\
Q_h       &= Q_{nom} \times 0.00   =   0.00   \text{ KN}
\end{align*}

Αφόρτιστη γερανογέφυρα
\begin{align*}
\sum{Q_{r,min}}   &= \dfrac{G_{cr,φ}}{2} + \dfrac{G_{tr,φ} \times (L - a)}{L} =  16.29 \text{ KN} \\
Q_{r,min}         &= \dfrac{\displaystyle\sum{Q_{r,min}}}{2}                  =   8.14  \text{ KN} \\
\sum{Q_{r,(min)}} &= \dfrac{G_{cr,φ}}{2} + \dfrac{G_{tr,φ} \times a}{L}       =  13.78 \text{ KN} \\
Q_{r,(min)}       &= \dfrac{\displaystyle\sum{Q_{r,(min)}}}{2}                =   6.89  \text{ KN}
\end{align*}

Φορτισμένη γερανογέφυρα
\begin{align*}
\sum{Q_{r,\max}}     &= \dfrac{G_{cr,φ}}{2} + \dfrac{(G_{tr,φ} + Q_h) \times (L - a)}{L}   =  16.29 \text{ KN} \\
Q_{r,\max}           &= \dfrac{\displaystyle\sum{Q_{r,\max}}}{2}                           =   8.14  \text{ KN} \\
\sum{Q_{r,(\max)}}   &= \dfrac{G_{cr,φ}}{2} + \dfrac{(G_{tr,φ} + Q_h) \times a}{L}         =  13.78 \text{ KN} \\
Q_{r,(\max)}         &= \dfrac{\displaystyle\sum{Q_{r,(min)}}}{2}                          =   6.89 \text{ KN}
\end{align*}

\subsubsection{4ος συνδυασμός φορτίων}
Δυναμικά  φορτία
\begin{align*}
G_{cr,φ}  &= G_{cr} \times φ_4    =  27.27 \text{ KN} \\
G_{tr,φ}  &= G_{tr} \times φ_4    =   2.80 \text{ KN} \\
Q_h       &= Q_{nom} \times φ_4   =  32.00   \text{ KN}
\end{align*}

Αφόρτιστη γερανογέφυρα
\begin{align*}
\sum{Q_{r,min}}   &= \dfrac{G_{cr,φ}}{2} + \dfrac{G_{tr,φ} \times (L - a)}{L} =  16.29 \text{ KN} \\
Q_{r,min}         &= \dfrac{\displaystyle\sum{Q_{r,min}}}{2}                  =   8.14  \text{ KN} \\
\sum{Q_{r,(min)}} &= \dfrac{G_{cr,φ}}{2} + \dfrac{G_{tr,φ} \times a}{L}       =  13.78 \text{ KN} \\
Q_{r,(min)}       &= \dfrac{\displaystyle\sum{Q_{r,(min)}}}{2}                =   6.89  \text{ KN}
\end{align*}

Φορτισμένη γερανογέφυρα
\begin{align*}
\sum{Q_{r,\max}}     &= \dfrac{G_{cr,φ}}{2} + \dfrac{(G_{tr,φ} + Q_h) \times (L - a)}{L}   =  46.63 \text{ KN} \\
Q_{r,\max}           &= \dfrac{\displaystyle\sum{Q_{r,\max}}}{2}                           =  23.31  \text{ KN} \\
\sum{Q_{r,(\max)}}   &= \dfrac{G_{cr,φ}}{2} + \dfrac{(G_{tr,φ} + Q_h) \times a}{L}         =  15.44 \text{ KN} \\
Q_{r,(\max)}         &= \dfrac{\displaystyle\sum{Q_{r,(min)}}}{2}                          =   7.72 \text{ KN}
\end{align*}

\subsubsection{5oς συνδυασμός φορτίων}Ομοίως με τον 4ο.

\subsubsection{6oς συνδυασμός φορτίων}Ομοίως με τον 4ο.

\subsection{Οριζόντια φορτία λόγω επιτάχυνσης ή επιβράδυνσης}
Η κινητήρια δύναμη της γερανογέφυρας $K$ ασκείται στον άξονα της κίνησής της. H δύναμη αυτή ισομοιράζεται σε δύο οριζόντιες κατά μήκος της τροχιάς δυνάμεις $H_L$ ασκούμενες στους τροχούς. Το κέντρο βάρους όμως του συστήματος γερανογέφυρα - αναρτημένο φορτίο, βρίσκεται σε απόσταση $l_s$ από τον άξονα της κίνησης, με αποτέλεσμα να παράγεται μια στρεπτική ροπή $M$. Για λόγους ισορροπίας λοιπόν, αναπτύσσεται το ζεύγος των αντίρροπων δυνάμεων $H_{T1}$ - $H_{T2}$ στις θέσεις των τροχών.

\subsubsection{1ος και 2ος συνδυασμός φόρτισης}
Για τους δύο πρώτους συνδυασμούς φόρτισης ισχύει ότι:
\begin{align*}
K         &= μ \times m_w \times Q_{r,(min)}  =   3.03 \text{KN} \\ 
H_{L,1}   &= \dfrac{φ_5 \times K} {n_r} =   2.27 \text{KN} \\ 
H_{L,2}   &= \dfrac{φ_5 \times K} {n_r} =   2.27 \text{KN} \\ 
ξ_1       &= \dfrac{\displaystyle\sum{Q_{r,max}}}{\displaystyle\sum{Q_{r,max} + \displaystyle\sum{Q_{r,(max)}}}} =   0.76 \text{KN} \\ 
ξ_2       &= 1 - ξ1 =   0.24 \text{KN} \\ 
l_s       &= (ξ_1 - 0.5) \times ξ_2 =   3.92 \text{KN} \\ 
M         &= K \times l_s =  11.87 \text{KN} \\ 
H_{T1}    &= φ_5 \times ξ_2 \times \dfrac{M}{a} =   2.66 \text{KN} \\ 
H_{T2}    &= φ_5 \times ξ_1 \times \dfrac{M}{a} =   8.47 \text{KN} 
\end{align*}

\subsubsection{3ος, 4ος και 8ος συνδυασμός φόρτισης}
Για τον 3ο, τον 4ο και τον 8ο συνδυασμό φόρτισης ισχύει ότι:
\begin{align*}
K         &= μ \times m_w \times Q_{r,(min)}  =   2.76 \text{KN} \\ 
H_{L,1}   &= \dfrac{φ_5 \times K} {n_r} =   2.07 \text{KN} \\ 
H_{L,2}   &= \dfrac{φ_5 \times K} {n_r} =   2.07 \text{KN} \\ 
ξ_1       &= \dfrac{\displaystyle\sum{Q_{r,max}}}{\displaystyle\sum{Q_{r,max} + \displaystyle\sum{Q_{r,(max)}}}} =   0.75 \text{KN} \\ 
ξ_2       &= 1 - ξ1 =   0.25 \text{KN} \\ 
l_s       &= (ξ_1 - 0.5) \times ξ_2 =   3.77 \text{KN} \\ 
M         &= K \times l_s =  10.38 \text{KN} \\ 
H_{T1}    &= φ_5 \times ξ_2 \times \dfrac{M}{a_r} =   2.42 \text{KN} \\ 
H_{T2}    &= φ_5 \times ξ_1 \times \dfrac{M}{a_r} =   7.31 \text{KN} 
\end{align*}




\subsection{Οριζόντια φορτία λόγω παράγωγης κίνησης της γερανογέφυρας (συνδυασμός 5)}

Κατά την κίνηση της η γερανογέφυρα ακολουθεί μια λοξή πορεία ως προς τις τροχιές της. Το γεγονός
αυτό έχει ως συνέπεια τη δημιουργία των εξής δυνάμεων:

\begin{itemize}
    \item της οδηγούσας δύναμης $S$ που ασκείται στη θέση των μέσων καθοδήγησης, και
    \item των εγκάρσιων ως προς τις τροχιές δυνάμεων που ασκούνται σε κάθε τροχό.
\end{itemize}

\begin{align*}
    \sum{Q_r} &= \displaystyle\sum{Q_{r,max}} + \displaystyle\sum{Q_{r,(max)}} = {{ "%.3f" % SQr }} \text{ KN} \\
    \xi_1     &= \dfrac{\displaystyle\sum{Q_{r,max}}}{\displaystyle\sum{Q_{r,max}} + \displaystyle\sum{Q_{r,(max)}}} = {{ "%.3f" % ksi1 }} \\
    \xi_2     &= 1 - \xi1 = {{ "%.3f" % ksi2 }}\\
    \alpha    &= {{ "%.4f" % a_rad }} \text{ rad} \\
    f         &= 0.3 \cdot \left(1 - e^{-250 \cdot \alpha}\right) = {{ "%.3f" % f }}
\end{align*}

Αφού το σύστημα κύλισης της γερανογέφυρας είναι IFF έχουμε:
\begin{align*}
    h                &= \dfrac{m \cdot \xi_1 \cdot \xi_2 \cdot L^2 + \sum{e_i^2}}{\sum{e_i}} = {{ "%.3f" % h }} \\
    \lambda_s        &= 1 - \dfrac{\sum{e_i}}{n \cdot h} =  {{ "%.3f" % l_s }} \\
    \lambda_{S,11,L} &= {{ "%.3f" % l_s11L }} \\
    \lambda_{S,12,L} &= {{ "%.3f" % l_s12L }} \\
    \lambda_{S,21,L} &= {{ "%.3f" % l_s21L }} \\
    \lambda_{S,22,L} &= {{ "%.3f" % l_s22L }} \\
    \lambda_{S,11,T} &= \dfrac{\xi_2}{n} \cdot \left(1 - \dfrac{e_1}{h}\right) = {{ "%.3f" % l_s11T }} \\
    \lambda_{S,12,T} &= \dfrac{\xi_2}{n} \cdot \left(1 - \dfrac{e_2}{h}\right) = {{ "%.3f" % l_s12T }} \\
    \lambda_{S,21,T} &= \dfrac{\xi_1}{n} \cdot \left(1 - \dfrac{e_1}{h}\right) = {{ "%.3f" % l_s21T }} \\
    \lambda_{S,22,T} &= \dfrac{\xi_1}{n} \cdot \left(1 - \dfrac{e_2}{h}\right) = {{ "%.3f" % l_s22T }}
\end{align*}

Άρα οι ασκούμενες δυνάμεις είναι:
\begin{align*} %
    S                &= f \cdot \lambda_s \cdot \sum{Q_r} = {{ "%.3f" % S }} \text{ KN} \\
    H_{S,11,L}       &= f \cdot \lambda_{s,11,L} \cdot \sum{Q_r} = {{ "%.3f" % H_s11L }} \text{ KN} \\
    H_{S,12,L}       &= f \cdot \lambda_{s,12,L} \cdot \sum{Q_r} = {{ "%.3f" % H_s12L }} \text{ KN} \\
    H_{S,21,L}       &= f \cdot \lambda_{s,21,L} \cdot \sum{Q_r} = {{ "%.3f" % H_s21L }} \text{ KN} \\
    H_{S,22,L}       &= f \cdot \lambda_{s,22,L} \cdot \sum{Q_r} = {{ "%.3f" % H_s22L }} \text{ KN} \\
    H_{S,11,T}       &= f \cdot \lambda_{s,11,T} \cdot \sum{Q_r} = {{ "%.3f" % H_s11T }} \text{ KN} \\
    H_{S,12,T}       &= f \cdot \lambda_{s,12,T} \cdot \sum{Q_r} = {{ "%.3f" % H_s12T }} \text{ KN} \\
    H_{S,21,T}       &= f \cdot \lambda_{s,21,T} \cdot \sum{Q_r} = {{ "%.3f" % H_s21T }} \text{ KN} \\
    H_{S,22,T}       &= f \cdot \lambda_{s,22,T} \cdot \sum{Q_r} = {{ "%.3f" % H_s22T }} \text{ KN}
\end{align*}

Τελικά στη θέση του προπορευόμενου ζεύγους τροχών έχουμε:
\begin{align*}
    H_{S,1T} &= H_{S,11,T} - S = {{ "%.3f" % H_s1T }} \text{ KN} \\
    H_{S,2T} &= H_{S,21,T} = {{ "%.3f" % H_s21T }} \text{ KN}
\end{align*}
