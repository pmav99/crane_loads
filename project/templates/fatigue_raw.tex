\section{Φορτίο κόπωσης}
Για τον υπολογισμό των επιπτώσεων της κόπωσης χρησιμοποιείται ένα ισοδύναμο φορτίο κόπωσης. Για τον υπολογισμό του χρησιμοποιούνται οι χαρακτηριστικές τιμές των φορτίων της γερανογέφυρας (χωρίς δυναμικούς συντελεστές). Έτσι λοιπόν η μέγιστη χαρακτηριστική τιμή του φορτίου τροχού  $Q_{max,i}$ είναι:
\begin{align*}
\sum{Q_{max,i}}     &= \dfrac{G_{cr}}{2} + \dfrac{(G_{tr} + Q_h) \times (L - a)}{L}   = SQmax_i-- \text{ KN} \\
Q_{max,i}           &= \dfrac{\displaystyle\sum{Q_e}}{2}                               = Qmax_i--  \text{ KN}
\end{align*}

Ο δυναμικός συντελεστής της κόπωσης υπολογίζεται ως ακολούθως:
\begin{equation*}
φ_{fat} = \max \left( \dfrac{1+φ_1}{2}\ ; \ \dfrac{1+φ_2}{2} \right) =vfat--
\end{equation*}

Οι συντελεστές $λ_σ$ και $λ_τ$ εξαρτώνται από την κατηγορία κόπωσης της γερανογέφυρας. Έτσι σύμφωνα με τον πίνακα 2.12 του EC1991-3:2006 έχουμε ότι για κατηγορία κόπωσης $fatigue_class--$ ισχύει ότι:
\begin{align*}
λ_σ	&= lfat_s--\\
λ_τ	&= lfat_t--
\end{align*}

Τελικά το φορτίο τροχού για σχεδιασμό έναντι κόπωσης είναι:
\begin{align*}
Q_{e,σ} &= φ_{fat} \times λ_σ \times Q_{max,i} = Qes--\\
Q_{e,τ}	 &= φ_{fat} \times λ_τ \times Q_{max,i} = Qet--
\end{align*}
